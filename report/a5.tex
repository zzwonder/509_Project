\documentclass[12pt]{article}
\usepackage{amsmath, amssymb, amsthm}
\headheight 0pt
\usepackage{color}
\usepackage{verbatim}
\usepackage{listings}
\usepackage{amsmath, amssymb}
\usepackage{listings}
\usepackage{xcolor}
\setlength{\textwidth}{6.25in}          
\setlength{\oddsidemargin}{0in}   
\setlength{\evensidemargin}{.25in}  
\newcommand{\rarrow}{\rightarrow}
\newcommand{\darrow}{\leftrightarrow}
\newcommand{\logequiv}{{\models =\!\!\!|}}
\renewcommand{\phi}{\varphi}
\begin{document}
	\pagestyle{empty}
	
	\begin{center}
		{\Large COMP 409\\
			Assignment No. 5}\\
		\bf{Caihua Li and Zhiwei Zhang}\\
		{\bf Due date: November 8, 2018}
	\end{center}
	
	\noindent\emph{Jointly}: x hours
	
	\noindent \emph{Caihua Li (S01299500)}: y hours
	
	\noindent\emph{Zhiwei Zhang (S012990315)}:  z hours
	
	\bigskip
	
	\bigskip
	
	\noindent
	
	
	
	\bigskip
	
	\noindent
	{\bf Note}: The pages below refer to the text from the book by
	Enderton (pdf posted).
	
	\begin{enumerate}
		\item
		Exercises 1, 3-6 on p. 78.
		
		1)
		\begin{enumerate}
			\item $\forall x(N(x)\rightarrow <(0,x))$
			\item $(\forall x(N(x)\rightarrow I(x)))\rightarrow I(0)$ or $\neg(\forall x(N(x)\rightarrow(\neg I(x))))\rightarrow I(0)$
			\item $\forall x(N(x)\rightarrow\neg <(x,0))$
			\item $\forall x((\neg I(x)\rightarrow(\forall y(<(y,x)\rightarrow I(y))))\rightarrow I(x))$
			\item $\forall x(N(x)\rightarrow\neg(\forall y(N(y)\rightarrow <(y,x))))$
			\item $\forall x(N(x)\rightarrow\neg(\forall y(N(y)\rightarrow\neg <(y,x))))$
		\end{enumerate}
		
		3) 
		$$
			(\forall x. (E(x)\to A(x))) \to ((\forall y. (E(y))\to (\exists z. (A(z)\wedge (hy\approx hz))))
		$$
		
		4)
		\begin{enumerate}
			\item $\neg\forall x(P(x)\rightarrow\neg\forall y(T(y)\rightarrow Fxy))$
			\item $\forall x(P(x)\rightarrow\neg\forall y(T(y)\rightarrow\neg Fxy))$
			\item $\neg\forall x(P(x)\rightarrow\forall y(T(y)\rightarrow Fxy))$
		\end{enumerate}
		
		5)
		
		a) 
		$$
			\forall x. (J(x)\to \neg Dax)
		$$
		
		b) Here we understand "can't do any" as "can't do every"
		$$
			\exists x. (J(x)\wedge \neg Dax)
		$$
		6) $\forall x(\neg\forall y(Lxy))$
		
		
		\item
		Exercises 1-6 on pp. 94-95.
		
		1)Z 
		 
		a)
		
		$"\Rightarrow"$: Suppose $\Gamma;\alpha\models \phi$. We have $ model(\Gamma;\alpha)=model(\Gamma)\cap model(\alpha)\subseteq model(\phi)$
		
		 For every $(A,\tau)\in model(\Gamma)$,
		 
		 \begin{itemize}
		 	\item If $(A,\tau)\in \mathbb{U}-model(\alpha)$, where $\mathbb{U}$ is the universal set. Then we have $(A,\tau)\in((\mathbb{U}-model(\alpha))\cup model(\phi))=model(\alpha\to\phi)$, thus $\Gamma\models(\alpha\to\phi)$.
		 	\item Else if $(A,\tau)\in models(\alpha)$, then $(A,\tau)\in model(\Gamma)\cap model(\alpha)\subseteq model(\phi)\subseteq ((\mathbb{U}-model(\alpha))\cup model(\phi))=model(\alpha\to\phi)$, thus $\Gamma\models(\alpha\to\phi)$. 
		 \end{itemize}		
		
		
		$"\Leftarrow"$:
		Suppose $\Gamma\models(\alpha\to\phi)$,	
		we have $model(\Gamma)\subseteq model(\alpha\to\phi)$, which means $model(\Gamma)\subseteq ((\mathbb{U}-model(\alpha))\cup model(\phi))$, where $\mathbb{U}$ is the universal sat. 
		
		For every $(A,\tau)\in model(\Gamma;\alpha)=model(\Gamma)\cap model(\alpha)$. We have $(A,\tau)\in model(\Gamma)$ and $ (A,\tau)\in model(\alpha)$.
		
		Since $(A,\tau)\in model(\Gamma)$, we have 
		$$
		(A,\tau)\in ((\mathbb{U}-model(\alpha))\cup model(\phi))
		$$
		\begin{itemize}
			\item If  $(A,\tau)\in model(\phi)$, we have done because that means $model(\Gamma;\alpha)\subseteq model(\phi)$ and thus $\Gamma;\alpha\models \phi$.
			\item Else if $(A,\tau)\in (\mathbb{U}-model(\alpha))$, this case is impossible because we have $(A,\tau)\in model(\alpha)$
		\end{itemize}  
		
		b) Suppose we have $\phi \logequiv \psi$. Since $\phi\models \psi$, from a) we have $\models (\phi\to\psi)$. Similiarly we have $\models(\psi\to\phi)$. Therefore $\models(\phi\leftrightarrow\psi)$ holds.
		
		2) L
		
		3) for all 
		$
			(A,\tau)\models \{\forall x.(\alpha\to\beta),\forall x.\alpha\}
		$ we have
		
		$$(A,\tau)\models \{\forall x.(\alpha\to\beta)\}
		$$
		and
		$$(A,\tau)\models \{\forall x.\alpha\}.
		$$
		which mean
		$$(A,\tau[x\to a])\models (\alpha\to\beta) \,for\,all\,a
		$$
		and 
		$$(A,\tau[x\to b])\models \alpha \,for\,all\,b
		$$
		
		Thus for all $c\in D$,
		$$(A,\tau[x\to c])\models (\alpha\to\beta) 
		$$
		and 
		$$(A,\tau[x\to c])\models \alpha 
		$$
		
		Therefore for all $c\in D$,
		$$(A,\tau[x\to c])\models\beta$$
		
		which is equalivant to
		
		$$(A,\tau)\models\forall x.\beta$$
		4) Given a formula $\alpha$, define $$models(\alpha)=\{(A,a)\ |\ A,a\models\alpha\}$$ for every structures $A$ and every variable assignment $a$.\newline
		Therefore, for every $(A,a)\in models(\alpha)$, $A,a\models\alpha$.\newline
		According to relevance lemma, since $x\notin FVars(\alpha)$, which means that, for every value b, $a|_{FVars(\alpha)}=a[x\rightarrow b]|_{FVars(\alpha)}$ it follows that $$A,a[x\rightarrow b]\models\alpha$$ which is equivalent to $$A,a\models\forall x\alpha$$
		So, we conclude that $$\alpha\models\forall x\alpha$$
		
		
		5) For all $x,y$, assume $(x\approx y)$ holds, then 
		$$(fx\approx fy)$$ is true and 
		$$(Pzfx\leftrightarrow Pzfy)$$ holds and $(Pzfx\leftrightarrow Pzfy)$ is true. Therefore $$(x\approx y)\to Pzfx\to Pzfy$$ is valid.
		
		
		6) Part 1: if $theta$ is valid, then $\forall x\theta$ is valid.\newline
		
		Part 2: if $\forall x\theta$ is valid, then $\theta$ is valid.\newline
		
		
		\item
		Exercises 8-12 on p. 95.
		
		8) 
		\begin{itemize}
			\item "$\Rightarrow$": Assume $\models_{\mathfrak{U}}\tau$ holds, in other words, $\mathfrak{U}\models \tau$. Proof by contradiction, if $\Sigma\models \neg\tau$, since $\mathfrak{U}\models\Sigma$, then $\mathfrak{U}\models \neg\tau$, which is a conflict. Since either $\Sigma\models\tau$ or $\Sigma\models\neg\tau$ holds, $\Sigma\models\tau$ is true.
			\item Assume $\Sigma\models\tau$, since $\mathfrak{U}\models \Sigma$ we have $\models_{\mathfrak{U}}\tau$
		\end{itemize}
		$\Box$
		
		9) L
		
		10) Z
		
		$\models_{\mathfrak{U}}\forall v_2 Qv_1 v_2[c^{\mathfrak{U}}]$
		
		$\Leftrightarrow \mathfrak{U},\tau[v_1\to c^{\mathfrak{U}}]\models\forall v_2 Qv_1 v_2$
		
		$\Leftrightarrow \mathfrak{U},\tau[v_1\to c^{\mathfrak{U}},v_2\to a]\models Q v_1 v_2$ for all $a$
		
		$\Leftrightarrow \mathfrak{U},\tau[v_2\to a]\models Qcv_2$ for all $a$
		
		$\Leftrightarrow \mathfrak{U},\tau\models \forall v_2 Qcv_2$ 
		
		$\Leftrightarrow \models_{\mathfrak{U}} \forall v_2 Qcv_2$ (The right side is a sentence)
		
		 
		11) L
		
		12) 
		
		a) $\exists x.(y\approx x\cdot x)$
		
		b) $(x\cdot x\approx x+x) \wedge (\neg( x\cdot x\approx x)) $
		
		c) 
		We can represent the union of intervals whose endpoints are algebraic by a inequality.
		
		Suppose the unioned intervals are $[a_1,b_1],...,[a_k,b_k]$ where all $a_i$s and $b_i$s are algebraic. Then inequality:
		
		$$
		\prod_{i=1}^k((x+(-a_i))\cdot(x+(-b_i)))\le0
		$$
		
		define the union of intervals.
		
		Since $+,\cdot, \approx,0$ are already in the vocabulary of the languate, things left are how to define the algebraic real numbers $a_i$s and $b_i$s, how to define the binary relation $\le $ and unary function $-$. 
		
		\begin{itemize}
			\item \textbf{neg(-)}: unary function $-$ is defined as $-(x)=y$ iff $y+x=0$
			\item  \textbf{less than($\le $)}: binary relation $\le $ is defined as $x\le y$ iff $\exists w. \exists z.(( x+w\approx y)\wedge (w\approx z\cdot z))$
			\item \textbf{algebraic real numbers}: Since  $a_i$s and $b_i$s are algebraic, WLOG for $a_i$ there exists an integer-coefficient polynomial $p(x)$ where $a_i$ is a zero point. Suppose $p(x)$ has $n$ distinct real zero points and $a_i$ is the $lth$ smallest one, then $a_i$ can be defined as the assignment of $x_l$ which satisfied the following formula:
			
			$$
			\exists x_1,...,x_{n}. \bigwedge_{u\not=m}(\neg(x_u=x_m))\wedge (\bigwedge_{j=1}^{n}(p(x_j)\approx 0)\wedge
			\bigwedge_{q=1}^{n-1}(x_q\le x_{q+1})
			$$
			
			From Problem 11, we know that we are able to define all  integers by nesting successor relation. Thus we are able to define all the algebraic numbers. 
		\end{itemize}
		 Therefore, the union of intervals with algebraic endpoints are definable in  $\mathfrak{R}$.
		 
		 
		{\sc Notation}:
		\begin{itemize}
			\item
			$\Gamma;\alpha$ means $\Gamma\cup\{\alpha\}$
			\item
			$\models_A \varphi$ means $A\models\varphi$
			\item
			$\models_A \varphi(x)[a]$ means $A,[x\mapsto a] \models\varphi(x)$.
			\item
			$|A|$ refers to the domain of $A$.
		\end{itemize}
		\item
		Z Exercise 17(a) on p. 96.
		
		\textbf{Proof Sketch.} We construct a formula $\phi$ which charactrizes the struct of $\mathfrak{U}$ thus it is satisfiable in $\mathfrak{U}$. Since in $\mathfrak{B}$, $\phi$  is also satisfiable, the assignment in struct $\mathfrak{B}$ will give us the bijection as isomorphism.
		
		\textbf{Proof.} We construct a formula $\phi$ which characterize the struct of $\mathfrak{U}$ as follows:
		
		variables of $\phi$: $x_1,x_2,...,x_{|D^\mathfrak{U}|+1}$
		
		Free variables of $\phi$: $x_1,x_2,...,x_{|D^\mathfrak{U}|}$
		
		$$
	\phi=\bigwedge_{\substack{i\neq j;\\1\le i,j\le|D_\mathfrak{U}|}}(x_i\not \approx x_j)\wedge \bigwedge_{\substack{i\neq j;\\1\le i,j\le|D_\mathfrak{U}|}}\psi_{ij} \wedge (\neg \exists x_1,...,x_{|D^\mathfrak{U}+1|}. \bigwedge_{{\substack{i\neq j;\\1\le i,j\le|D_\mathfrak{U}|+1}}}(x_i\not \approx x_j))
		$$

  where atom formula $\psi_{ij}$ is defined as:
  $$\psi_{ij}=
  \begin{cases}
  	Px_ix_j, \,if\,(c^{\mathfrak{U}}_i,c^{\mathfrak{U}}_j)\in P^{\mathfrak{U}}\\
  	\neg	Px_ix_j, \,if\,(c^{\mathfrak{U}}_i,c^{\mathfrak{U}}_j)\not \in P^{\mathfrak{U}}
  \end{cases}		
  $$
  
  Since $\mathfrak{U}$ is finite, the formula $\phi$ is well defined.
  
  It clear that $\phi$ is satisfiable in $\mathfrak{U}$ by assigning $\tau(x_i)=c_i^{\mathfrak{U}}$ for every $x_i$ with $1\le i\le D^{|\mathfrak{U}|}$. Since $\mathfrak{U}\equiv\mathfrak{B}$, $\phi$ must be satisfiable in $\mathfrak{B}$, which means there exists an assignment $\tau: FV(\phi)\to D^{\mathfrak{B}}$ s.t. $(\mathfrak{B},\tau)\models\phi$. Let $I:D^{\mathfrak{U}}\to FV(\phi)$ be the bijection s.t. $I(x_i)=c^{\mathfrak{U}}_i$ and  $\tau'= \tau\cdot I$ be the composition of $I$ and $\tau$. We have $\tau':D_{\mathfrak{U}}\to D_{\mathfrak{B}}$
  
  
  Next we will prove $\tau$ is the bijection from $D_{\mathfrak{U}}$ to $D_{\mathfrak{B}}$ s.t.:
  
  $$P^{\mathfrak{U}}(c_i^{\mathfrak{U}},c_j^{\mathfrak{U}})\,\Leftrightarrow\,P^{\mathfrak{B}}(\tau'(c_i^{\mathfrak{U}}),\tau'(c_j^{\mathfrak{U}}))$$
  
  which means $\mathfrak{U}$ is isomorphic to $\mathfrak{B}$.
  
  \textbf{bijection:} It's easy to see $I$ is a bijection. Thus we only need to prove $\tau$ is a bijection from $FV(\phi)$ to $D^{\mathfrak{B}}$. The subformula $\bigwedge_{\substack{i\neq j;\\1\le i,j\le|D_\mathfrak{U}|}}(x_i\not \approx x_j)$ of $\phi$ guarantees that $\tau$ is one-to-one. And subformula $\bigwedge_{\substack{i\neq j;\\1\le i,j\le|D_\mathfrak{U}|}}(x_i\not \approx x_j)$ and \\$(\neg \exists x_1,...,x_{|D^\mathfrak{U}+1|}. \bigwedge_{{\substack{i\neq j;\\1\le i,j\le|D_\mathfrak{U}|+1}}}(x_i\not \approx x_j)) $ together make $|D^{\mathfrak{B}}|=|D^{\mathfrak{U}}|=|FV(\phi)|$, which means $\tau$ is onto. Thus $\tau$ is a bijection.
  
  \textbf{maintenance of P on $\tau'$:} 
  for all $1\le i,j\le |D^{\mathfrak{U}}|$ we have:
  
  $P^{\mathfrak{B}}(\tau'(c_i^{\mathfrak{U}}),\tau'(c_j^{\mathfrak{U}}))$\\
  $\Leftrightarrow P^{\mathfrak{B}}(\tau(I(c_i^{\mathfrak{U}})),\tau I(c_j^{\mathfrak{U}})))$\\
    $\Leftrightarrow P^{\mathfrak{B}}(\tau(x_i),\tau( x_j))$\\
    $\Leftrightarrow \mathfrak{B},\tau\models P(x_i,x_j)$\\
     $\Leftrightarrow P^{\mathfrak{U}}(c_i^{\mathfrak{U}},c_j^{\mathfrak{U}}))$ ($\mathfrak{B},\tau\models \bigwedge_{\substack{i\neq j;\\1\le i,j\le|D_\mathfrak{U}|}}\psi_{ij}$)
     
     Therefore, $\mathfrak{B}$ and $\mathfrak{U}$ are isomorphic.
		\item
		L Consider the following English sentences:
		\begin{itemize}
			\item
			``There are some critics who admire only one another.''
			\item
			``It is not the case that there are some numbers among which none is
			least''.
		\end{itemize}
		Can you formalize these sentences in first-order logic? How?
		\item
		ZL Show that the following formulas are valid,
		where in (b)-(i) $x$ is not free in $\beta$.
		Can the material implication ia (a) be reversed?
		\begin{enumerate}
			\item
			$\forall x (\alpha \rightarrow \beta) \rightarrow
			(\forall x \alpha \rightarrow \forall x \beta)$
			\item
			$\forall x (\alpha \wedge \beta) \leftrightarrow
			(\forall x \alpha \wedge \beta)$
			\item
			$\exists x (\alpha \wedge \beta) \leftrightarrow
			(\exists x \alpha \wedge \beta)$
			\item
			$\forall x (\alpha \vee \beta) \leftrightarrow
			(\forall x \alpha \vee \beta)$
			\item
			$\exists x (\alpha \vee \beta) \leftrightarrow
			(\exists x \alpha \vee \beta)$
			
			\item
			$\forall x (\alpha \rightarrow \beta) \leftrightarrow
			(\exists x \alpha \rightarrow \beta)$
			
			For a struct $A$ and an assigment $\tau$:
			
			$(A,\tau)\models\forall x (\alpha \rightarrow \beta)$
			
		$\Leftrightarrow (A,\tau[x\to a])\models(\alpha \rightarrow \beta)$ for all $a$.
		
		$\Leftrightarrow (A,\tau[x\to a])\models(\neg \alpha) \,or\, (A,\tau[x\to a])\models \beta$ for all $a$.
		
		$\Leftrightarrow (A,\tau[x\to a])\models(\neg \alpha) \,or\, (A,\tau)\models \beta$ for all $a$ ($x\not\in FV(\beta)$).
		
		$\Leftrightarrow (A,\tau)\models\neg(\exists x \alpha) \,or\, (A,\tau)\models \beta$ 
		
			$\Leftrightarrow (A,\tau)\models(\exists x \alpha\to  \beta)$ 
		
			\item
			$\exists x (\alpha \rightarrow \beta) \leftrightarrow
			(\forall x \alpha \rightarrow \beta)$
			
			$(A,\tau)\models\exists x (\alpha \rightarrow \beta)$
			
			$\Leftrightarrow (A,\tau[x\to a])\models(\alpha \rightarrow \beta)$ for some $a$.
			
			$\Leftrightarrow (A,\tau[x\to a])\models(\neg \alpha) \,or\, (A,\tau[x\to a])\models \beta$ for some $a$.
			
			$\Leftrightarrow (A,\tau[x\to a])\models(\neg \alpha) \,or\, (A,\tau)\models \beta$ for some $a$ ($x\not\in FV(\beta)$).
			
			$\Leftrightarrow (A,\tau)\models\neg(\forall x \alpha) \,or\, (A,\tau)\models \beta$ 
			
			$\Leftrightarrow (A,\tau)\models(\forall x \alpha\to  \beta)$ 
			
			\item
			$\forall x (\beta \rightarrow \alpha) \leftrightarrow
			(\beta \rightarrow \forall x \alpha)$
			
			$(A,\tau)\models (\forall x.(\beta\to\alpha))$
			
			$\Leftrightarrow (A,\tau[x\to a])\models(\beta\to\alpha)$ for all $a$.
			
			$\Leftrightarrow (A,\tau[x\to a])\models\neg\beta$ or $\Leftrightarrow (A,\tau[x\to a])\models \alpha$ for all $a$
				
				$\Leftrightarrow (A,\tau)\models\neg\beta$ or $ (A,\tau[x\to a])\models \alpha$ for all $a$ ($x\not\in FV(\beta)$)
				
					$\Leftrightarrow (A,\tau)\models\neg\beta$ or $ (A,\tau)\models \forall x.\alpha$ 
					
						$\Leftrightarrow (A,\tau)\models(\beta \rightarrow \forall x \alpha)$ 
			\item
			$\exists x (\beta \rightarrow \alpha) \leftrightarrow
			(\beta \rightarrow \exists x \alpha)$
			
			$(A,\tau)\models \exists(\beta \rightarrow \alpha)$
			
			$\Leftrightarrow (A,\tau[x\to a])\models(\beta\to\alpha)$ for some $a$
			
			$\Leftrightarrow (A,\tau[x\to a])\models(\neg \beta)$ or $(A,\tau[x\to a])\models\alpha$ for some $a$
			
			$\Leftrightarrow (A,\tau)\models(\neg \beta)$ or $(A,\tau[x\to a])\models\alpha$ for some $a$	($x\not\in FV(\beta)$)
			
				$\Leftrightarrow (A,\tau)\models(\neg \beta)$ or $(A,\tau)\models\exists x.\alpha$	
				
					$\Leftrightarrow (A,\tau)\models (\beta\to\exists x\alpha)$
		\end{enumerate}
		\item
		Z Assume a relational vocabulary (i.e., no function symbols).
		For a sentence $\phi$ of 1st-order logic with equality,
		let $\phi'$ be the result of replacing every atomic formula
		$x=y$ in $\phi$ by $E(x,y)$, where $E$ is a new binary predicate
		symbol, and then conjoining with the equivalence and congruence
		axioms for $E$.
		
	
		
		(The equivalence axioms says that $E$ is reflexive, symmetric and
		transitive. The congurence axioms says that
		if $P(a_1,\ldots,a_k)$ holds and $E(a_i,b_i)$ holds
		for $i=1,\ldots,k$, then $P(b_1,\ldots,b_k)$ holds.)
		Show that $\phi$ is satisfiable iff $\phi'$ is satisfiable.
		(Recall that a sentence is satisdfiable if it is satisfied by some
		structure.) (Hint: You can use equivalence classes as elements.)
		
		
			\begin{itemize}
			\item "$\Rightarrow$": This is an easy direction. Suppose $\phi$ is satisfiable, then there exists $(A,\tau)$. Then consider  $A'$, which is obtained by adding the interpetation of $E^{A'}$ as the identity relationship $=$ . Note that identity is a binary relation with the equivalence and congruence axioms.  Also we can notice now $"="$ in $\phi$ and $"E"$ in $\phi'$ has the same interpetation  and so do other symbols. Thus $(A',\tau')\models \phi'$, which means $\phi'$ is satisfiable.
			\item "$\Leftarrow$": Since it is not obvious to use induction on sentences, we will prove a stronger result about formula, other than sentences. Then  the statement about sentence will be a corollary of it.
			
			\textbf{Lemma 1.} Let $\phi$ be an formula of relational vocabulary with predicate symbols and equality and $\phi'$ is obtained by replacing every atomic formula $x=y$ in $\phi$.
			Suppose $\phi'$ is satisfiable on $A_E$. Then for every $\tau_E$, 
			  $$(A_E,\tau_E)\models \phi'\, \Leftrightarrow (A_{=},\tau_=)\models\phi$$
			  where $A_{=},\tau_=$ is defined as:
			
			$D_{A_=}=\{e_1,...,e_k\}$ is consisted of the equivalence classes of $D_{A_E}$.
			
			For every relation $P_i^=$ with arity $k_i$,
			
			$P_i^{A=}(e_1,...,e_{k_i})$ iff $(\exists c_1,...,c_k\in D_{A_E}. P^E(c_1,...,c_k)\wedge \bigwedge_{j=1}^{k_i}(c_j\in e_j))$ (*)
			
			$\tau_=(x)=e(\tau_E(x))$, where $e(\tau_E(x))$ donotes the equivalent class of $\tau_E(x)$. \\
		
		
		If Lemma 1 holds, then $\phi'$ is satisfiable implies $\phi$ is satisfiable will follow. Since sentences are special formulas, the statement will hold. Next we will prove Lemma 1, which completes our proof.
			
			\bigskip
		\textbf{Proof of Lemma 1.}
		
		We will prove by induction on $\phi'$.
		
		\begin{itemize}
			\item Basis:
			\begin{itemize}
				\item $\phi'$ is $E(x,y)$. In this case $\phi$ is $x=y$. 
				\begin{itemize}
					\item If $A_E,\tau_E\models E(x,y)$. Then we know $\tau_E(x)$ and $\tau_E(y)$ are in the same equivalent class. Thus $e(\tau_E(x))=e(\tau_E(y))$ holds, which means $\tau_=(x)=\tau_=(y)$ and $A_=,\tau_=\models \phi $.
					\item If $e(\tau_E(x))=e(\tau_E(y))$, $\tau_E(x)$ and $\tau_E(y)$ are in the same equivalent class, then $A_E,\tau_E\models E(x,y)$ holds.
				\end{itemize}
				\item $\phi'$ is a $k_i$-arity predicate $P(x_1,...,x_{k_i})$.
				\begin{itemize}
					\item If $A_E,\tau_E\models P(x_1,...,x_{k_i})$, which means there exist $c_1,...,c_k\in D_{A_E}$ where $c_i=\tau_E(x)$ for every $i$, s.t. $P(c_1,...,c_k)$ is true. Thus $P_{A_=}(e(c_1),...,e(c_k))$ from (*). Since $\tau_=(x)=e(\tau_E(x))$, that means $A_=,\tau_=\models \phi$ 
					\item If $A_=,\tau_=\models P(x_1,...,x_k)$. Then by (*) we know there exist $c_1,...c_k \in D_{A_E}$ s.t. $P^E(c_1,...,c_k)$ and $e(c_j)=e(\tau_E(x))$, in other words, $E(c_j,\tau_E(x))$ is true.
					Since we have congruence axiom for $E$, $P^E(\tau_E(x_1),...,\tau_E(x_k))$ holds, which means $A_E,\tau_E\models \phi'$.
				\end{itemize} 
			\end{itemize}
		\item Inductive Step: (For simplicity we just consider binary connective $\wedge$ since $\{\neg,\wedge\}$ is adequate)
		\begin{itemize}
			\item $\phi'$ is $\neg \psi'$\\
				For all $\tau_E$, we have\\
				$A_E,\tau_E\models \phi'$\\
				$\Leftrightarrow A_E,\tau_E\not\models \psi'$\\
				$\Leftrightarrow A_=,\tau_=\not\models \psi$ (I.H.)\\
				$\Leftrightarrow A_=,\tau_=\models \neg\psi$ \\
				$\Leftrightarrow A_=,\tau_=\models \phi$
			\item $\phi'$ is $\psi'_1\circ\psi'_2$\\
			For all $\tau_E$ we have\\
				$A_E,\tau_E\models \phi'$\\
			$\Leftrightarrow A_E,\tau_E\models\psi'_1\wedge\psi'_2$\\
			$\Leftrightarrow A_=,\tau_=\models\psi_1$ and $ A_=,\tau_=\models\psi_2$ (I.H.)\\
			$\Leftrightarrow A_=,\tau_=\models \psi_1\wedge\psi_2$ \\
			$\Leftrightarrow A_=,\tau_=\models \phi$
			\item $\phi'$ is $\exists x. \psi'$\\
			For all $\tau_E$ we have\\
				$A_E,\tau_E\models \phi'$\\
					$\Leftrightarrow A_E,\tau_E\models \exists x.\psi'$\\
						$\Leftrightarrow A_E,\tau_E[x\to a]\models \psi'$ for some $a\in D_{A_E}$\\
							$\Leftrightarrow A_=,\tau_=[x\to e(a)]\models \psi$ for some $a\in D_{A_E}$ (I.H.)\\
								$\Leftrightarrow A_=,\tau_=[x\to b]\models \psi$ for some $b\in D_{A_=}$\\
								$\Leftrightarrow A_=,\tau_=\models \exists x.\psi$\\
									$\Leftrightarrow A_=,\tau_=\models \phi$\\
			\item $\phi'$ is $\forall x. \psi'$
				$A_E,\tau_E\models \phi'$\\
				$\Leftrightarrow A_E,\tau_E\models \forall x.\psi'$\\
				$\Leftrightarrow A_E,\tau_E[x\to a]\models \psi'$ for all $a\in D_{A_E}$\\
				$\Leftrightarrow A_=,\tau_=[x\to e(a)]\models \psi$ for all $a\in D_{A_E}$ (I.H.)\\
				$\Leftrightarrow A_=,\tau_=[x\to b]\models \psi$ for all $b\in D_{A_=}$\\
				$\Leftrightarrow A_=,\tau_=\models \forall x.\psi$\\
				$\Leftrightarrow A_=,\tau_=\models \phi$\\
		\end{itemize}
		\end{itemize}
		$\Box$ 
		\end{itemize}
	
		\item
		L An {\em existential-conjunctive\/} formula is a formula of the form
		$(\exists x_1) \ldots (\exists x_n) \bigwedge_{i=1}^k \alpha_i$,
		where each $\alpha_i$ is an atomic formula. What is the complexity
		(data and query complexity) of evaluating existential-conjunctive queries? 
		(Focus on upper bounds.)
		
		Consider a vocabulary of one binary relation symbol ${\bf R}$.
		Let $A=(D,R)$ be a structure with $D=\{1,2,3\}$ and
		$R=\{\langle 1,2\rangle,
		\langle 2,1\rangle,
		\langle 1,3\rangle,
		\langle 3,1\rangle,
		\langle 2,3\rangle,
		\langle 3,2\rangle\}$.
		With each graph $G=(V,E)$ we associate a sentence $\varphi_G$ as
		follows. Let $V=\{v_1,\ldots,v_n\}$. Then $\varphi_G$ is
		$$ (\exists x_1) \ldots (\exists x_n) 
		\bigwedge_{(v_i,v_j)\in E} R(x_i,x_j).$$
		(Note that $\varphi_G$ is an existential-conjunctive formula.)
		Show that $A \models \varphi_G$ iff $G$ is 3-colorable.
		What can you conclude from this about the complexity
		of evaluating existential-conjunctive queries? 
		(Discuss upper and lower bounds.)
		\item
		Z {\sf Drinker's Principle}: ``In every group of people one can
		point to one person in the group such that if that person drinks then
		all the people in the group drink.''
		
		Formulate this principle in first-order logic and prove its validity.
		
		\textbf{Solution:}
		
		$$
		\exists x. (D(x)\to(\forall y.(D(y))))
		$$
		
		Proof of validity:
		
		for all $(A,\tau)$,
		
		$$
		(A,\tau)\models \exists x. (D(x)\to(\forall y.(D(y))))
		$$
		
		iff 
		
		$$
		(A,\tau[x\to a])\models  (D(x)\to(\forall y.(D(y))))
		$$
		
		for some $a$
		
		iff
		
		$$
		(A,\tau[x\to a])\models \neg D(x)
		\,or\,
		(A,\tau)\models \forall y.(D(y))
		$$
		for some $a$
		
		If $(A,\tau[x\to a])\models \neg D(x)$ for some $a$ holds, then we have done. Else it means $(A,\tau[x\to a])\models D(x)$ for all $a$ is true, which is equivalent to $	(A,\tau)\models \forall y.(D(y))$ is true. $\Box$
	\end{enumerate}
\end{document}
